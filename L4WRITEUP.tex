%%%%% Document Setup %%%%%%%%

\documentclass[12pt, onecolumn]{revtex4}    % Font size (12pt) and column number (one or two).

\usepackage{times}                          % Times New Roman font type
\usepackage[dvipsnames]{xcolor}
\usepackage[a4paper, left=2.5cm, right=2.5cm,
 top=2.5cm, bottom=2.5cm]{geometry}       % Defines paper size and margin length
\usepackage{url}
\renewcommand{\baselinestretch}{1.15}     % Defines the line spacing

\usepackage[font=small,
labelfont=bf]{caption}                      % Defines caption font size and caption title bolded

\usepackage{graphics,graphicx,epsfig,ulem}	% Makes sure all graphics works
\usepackage{amsmath} 						% Adds mathematical features for equations

\usepackage{etoolbox}                       % Customise date to preferred format
\makeatletter
\patchcmd{\frontmatter@RRAP@format}{(}{}{}{}
\patchcmd{\frontmatter@RRAP@format}{)}{}{}{}
\renewcommand\Dated@name{}
\makeatother

\usepackage{fancyhdr}
\usepackage{ amssymb }

\pagestyle{fancy}                           % Insert header
\renewcommand{\headrulewidth}{0pt}
\lhead{\small Georgia Hills}                          % Your name
\rhead{\small Strong Coupling Constant Extraction at energy s = ${\sqrt 13}$ TeV}            % Your report title               

\def\thesection{\arabic{section}}

\def\bibsection{\section*{References}}        % Position reference section correctly


%%%%% Document %%%%%
\begin{document}                     


\title{Strong Coupling Constant Extraction} 
\date{Submitted: \today{}}
\author{Georgia Hills}
\affiliation{\normalfont Level 4 Project, MPhys Natural Sciences\\ Supervisor: Professor D. Ma\^itre\\ Department of Physics, Durham University}

\begin{abstract}              
 


\end{abstract}


\maketitle


\tableofcontents
\let\toc@pre\relax
\let\toc@post\relax

\newpage


\section{Introduction} 


The strong coupling constant ${\alpha _s}$ determines the magnitude of an interaction between particles with colour charge; quarks and gluons \cite{PPB}. It cannot be predicted from first principles, so must be extracted from experiments, and is a parameter of Quantum Chromodynamics (QCD) \cite{DMP}.  ${\alpha _s}$ was obtained from a ${\chi^2}$ test, see Section (\ref{stats}). 
Monte-Carlo simulations are computational algorithms that simulate events and generate data, and hence can be used to generate theoretical predictions of ${\alpha _s}$ \cite{MONTE}. These theoretical predictions were used to compare against data generated from experiments at the ATLAS collaboration. The data was measured at a centre of mass  energy of ${s = \sqrt13}$ TeV for multiple jet processes \cite{HEPP}. The data was generated by colliding two protons at high energies and measuring the outcome. As protons are hadrons they are composite objects; the centre of mass energy isn't fixed, thus a huge spectrum of energies can be considered. However the partons which are charged under the strong force and emit vast amounts of radiation, which gives rise to jets \cite{PHD}. The background of an experiment is often contributed to by these jets. Jets of Z + 2,3,4 differential cross section are used to extract ${\alpha _s}$ \cite{HEPP}. A jet is generated due to energetic gluons forming hadrons. Previous values of ${\alpha _s}$ have been extracted using similar methods and yielded a value of ${\alpha _s(M_z) = 0.1178^{+0.0051}_{-0.0043}}$ at a centre of mass energy of ${s=\sqrt7}$  TeV. Other values have been extracted by comparing experimental data from deep inelastic scattering,  hadronic ${\tau}$ decays, against predictions from the theory \cite{DMP}. ${\alpha _s}$ can be expanded using perturbation theory thus has different orders of sensitivity. In addition to this different experiments have different levels of precision, thus obtaining a value of ${\alpha_s}$ is a balance between these factors.The value of ${\alpha _s}$ presented here is at next-leading order (NLO) \cite{DMP}. 



\section{Theory} 

QCD is a non-abelian i.e. non hermitian, gauge field theory; the field is defined such that the transformation rules are only defined locally. A unitary matrix is a matrix whose transpose is the same as its' inverse \cite{BOOK}. QCD describes the strong interaction and is described by the ${SU(3)}$ part of the ${SU(3)\times SU(2) \times U(1)}$ Standard Model of Particle Physics (SM) \cite{PPB}. The SM is well-tested and one of the most predictive theories in the history of physics. However it is not complete as it fails to explain the existence of dark matter, or why there exists non-zero matter neutrinos. This is why measurements of constants is  key as in order to determine any discrepancies from the SM an accurate and well established of fundamental constants is required \cite{PHD}. The special unitary group ${SU(3)}$,  means that the representation of the group consists of unitary 3x3 matrices with determinant 1 \cite{GROUP}.  Whilst the ${SU(2) \times U(1)}$ part corresponds to the electroweak sector, whose interactions of the  ${W^\pm}$ and Z boson are mediated by the Higgs boson through electroweak symmetry breaking . These give rise to electromagnetism and radioactive decay \cite{PHD}. The Lagrangian of QCD is: \begin{equation} \mathcal{L} = \sum_ {q} \overline{\psi}_{q,a} (i\gamma^\mu \partial_\mu \delta_{ab} - g_s \gamma^\mu t_{ab}^C \mathcal{A}_\mu^C  - m_q\delta_{ab})\psi_{q,b} - \frac{1}{4} F_{\mu\nu}^A F^{A \mu\nu}, \end{equation} where the Einstein Summation Convention has been used. The colour-index, a, runs from 1 to  ${N_c = 3}$, ${m_q}$ represents the mass of the quarks and ${q}$ represents the quark flavour of the spinor $\psi_{q,a}$. Quarks are the fundamental representation i.e. an irreducible representation, of the ${SU(3)}$ group. The ${A_\mu^C}$ represents the gluon fields,  just as ${A_\mu}$ represents the photon field in electrodynamics. ${C}$ runs from 1 to ${N_c^2 -1 = 8}$, implying that there are 8 types of gluon. The ${t_{ab}^C}$ are the generators of the SU(3) group and are eight 3x3 matrices. When a quark and gluon interact the quark's colour is "rotated" in ${SU(3)}$ colour space, analogous to an object rotating in physical space, and it's rotation being described by a rotation matrix. The QCD coupling constant is given the symbol ${g_s}$. The field tensor ${F_{\mu\nu}^A}$ is defined: \begin{equation}F_{\mu\nu}^A  = \partial_\mu \mathcal{A}_\nu^A  - \partial_\nu \mathcal{A}_\mu^A - g_s f_{ABC} \mathcal{A}_\mu^B \mathcal{A}_\nu^C  \qquad [t^A,t^B] = if_{ABC}t^C,  \end{equation} where ${f_{ABC}}$ are the structure constants of the ${SU(3)}$ group \cite{PPB}.
A coupling constant determines how strong an interaction between two particles is when compared to the kinetic term within the Lagrangian of a process. The strong coupling constant ${\alpha _s}$ is not constant and varies at different energy scales. For low energy scales ${\alpha _s}$ is large and forces quarks to bind forming hadrons where perturbation theory cannot be used. For high energy scales ${\alpha _s}$ is small and thus quarks can interact with each other via scattering or annihilation to create new particles, and thus perturbation theory is valid. This is known as asymptotic freedom. Thus the strong force binds quarks together to form protons and neutrons in ordinary matter, and also binds protons together to form nuclei. ${\alpha _s}$ is said to be strong as it is of higher order than the magnitude of the electric charge, this leads to the hadronic theory of strong interactions thus non-perturbative methods have to be used, for low energy processes. The strong force is mediated by gluons just as the electromagnetic force is mediated by photons, the gluons carry a colour charge, similarly to the electric charge. There are three values of colour charge. Colour charge cannot be directly measured due to asymptotic freedom, i.e. one cannot measure quarks directly, and all hadrons are colour neutral. The need for colour charge can be explained by considering the ${\delta^{++}}$ baryon which consists of three up quarks (uuu). As a baryon is formed of three fermions the overall wave-function needs to be anti-symmetric under exchange of two fermions. However the spin, parity and spatial parts of the wave-function are all symmetric under exchange of two fermions (as they are identical), thus an additional quantum number is needed - colour charge.  In a collision of fermions there is a scattering centre of high energy, as particles propagate away from the centre they lose energy, ${\alpha _s}$ increases and binds quarks together, know as partons. The number of hadrons that are created outside of the centre is the number of jets, thus as the jets increase their likelihood decreases. Particles are always created in jets as the strong force means that overall stable matter must be colour neutral, this is known as colour confinement \cite{BOOK}. As we can only directly measure hadrons and not their constituents, partons, we must include a correction to the cross-section that signifies the correspondence between hadronic and partonic level \cite{DMP}. In this paper the process considered is the Drell-Yan process, where-in a quark of one hadron (proton) and an anti-quark of another hadron (proton) annihilate. This generates a Z-boson which subsequently decays into a lepton pair, either electron/positron ${e^+e^-}$ pair or muon/anti-muon ${\mu^+\mu^-}$ in this case. In the lowest energy process just the lepton pair is created, at higher energies more and more jets of hadrons are also created from gluon emission that decays into hadrons. 

\subsection{Factorisation, Renormalisation, and Next-Leading Order.} 

The factorisation and renormalisation scales were varied in order to calculate ${\alpha _s}$. The Factorisation scale essentially determines the "cut off point" between when perturbation theory is applicable and when it isn't. As  for higher energy ${\alpha _s}$ is small (perturbative) and for low energy ${\alpha _s}$ is large, due to asymptotic freedom as described above. Ideally this point should not matter as the same physics is being described. However when approximations are made and the sums do not contain an infinite number of terms the "cut off point" becomes relevant \cite{PHD}. Three values of factorisation are used, 0.5, 1, and 2. For each value one can calculate ${\alpha _s}$ and the difference between the values indicates the importance of those higher order terms not included in the summation. 
For high energy processes loop diagrams are included and this results in divergences which are known as Ultra-Violet (UV) divergences. Renormalisation is the procedure necessary to absorb these divergences and make the results physically realisable \cite{PHD}. This leads to ${\alpha _s}$ being a function of the unphysical renormalisation ${\mu^2}$, which is often taken close to the momentum transfer Q for a given process. The change of scale ${\mu}$ is only present in the next-leading order (NLO) of ${\alpha _s}$ so this scale is used.  This is known as the running coupling constant, and indicates the strength of the strong interaction in that process. NLO specifically means for a number of jets i, there are terms up to order ${\alpha _s^{i+1}}$. 

\subsection{Statistical Analysis and Extraction Procedure.} \label{stats}
In order to obtain a value of ${\alpha_s}$ the ${\chi^2}$ function must be minimised, a ${\chi^2}$ test shows the "goodness of fit" and determines the difference between experimental data and the theoretical predictions \cite{STAT}. When ${\chi^2}$ is minimised this is when the theory matches the experiment the most and hence corresponds to the best value of ${\alpha_s}$, which is denoted ${\alpha_0}$. ${\chi^2}$  is defined:  \begin{equation} \chi^2(\alpha_s(M_z)) = (y_t(\alpha_s(M_z))-y_d)^TC^{-1}(y_t(\alpha_s(M_z))-y_d), \end{equation} where ${y_t}$ are the predicted values of the bin from the theory, ${y_d}$ is the experimental values obtained at the LHC. The covariance matrix is calculated from the errors of different parts of the extraction and is defined \begin{equation} C = C_{exp} + C_{pdf} + C_{theory}. \end{equation} Each of these covariance matrices are calculated in different ways depending on their definition, ${C_{exp}}$ is the experiments error matrix, ${C_{pdf}}$ is the error associated with fitting a pdf to the data, and ${C_{theory}}$ is the error based on theoretical predictions. 

The experimental error was composed of three sources: statistical, luminosity, and systematic, which are all added together. The statistical error source  is fundamental to each measurement and is uncorrelated, hence the statistical component of the ${C_{exp}}$ matrix is diagonal and its elements are the statistical errors squared for each momentum bin \cite{DMP}. Luminosity describes how many particles there are in a given space in a given time: a higher luminosity means that particles are more likely to collide. The integrated luminosity was used, which is defined as the maximum luminosity and is the instantaneous number of interactions per second. A colliders aim is to optimise the integrated luminosity \cite{LUM}. The cross sections had a common uncertainty of 2.1 \% from the measurement of the integrated luminosity. The data generated at CERN had a total integrated luminosity of ${3.16   fb^{-1}}$ \cite{HEPP}. The systematic error is reducible and is introduced due to imprecision of instruments, or measurement process. The systematic and luminosity error is assumed to have "upward variation" for each source \cite{HEPD}. This means that the systematic and luminosty component of  ${C_{exp}}$, ${C}$, is calculated via: \begin{equation} \label{Cov} C_{ij}^{A} =  error^{A}_{i} \times error^{A}_{j}, \end{equation} where the ${^A}$ index indicates the type of systematic or luminosity error. Then each ${C_{ij}^{A}}$ are added together, this is then added to ${C_{exp}^{stat}}$ to form the full ${C_{exp}}$.

The pdf's used for this extraction were CT10nlo \cite{CT10}, CT14nlo \cite{CT14}, MSTW \cite{MSTW}, MMHT \cite{MMHT1, MMHT2}, NNPDF2.3 \cite{NN23}, NNPDF3.0 \cite{NN30}. The pdf's used were of two type; the "NN" pdf's are neural network pdf's which means they contain 100 replicas \cite{NN23}.  The values ${\alpha_s}$ for NNPDF2.3 were 0.114-0.124, and for NNPDF3.0 were 0.115, 0.117, 0.118, 0.119, 0.121. The other pdf's have a principle value and for this principle value there exists a set of 2nd approximation values of ${\alpha_s}$  up to a confidence value of 68\%. These pdf types also have sets for the non principle value of ${\alpha_s}$ around this principle value. All pdf sets were taken from LHAPDF \cite{LHAPDF}. The cross sections for the theoretical data, ${y_t(\alpha_s(M_z))}$, was generated for each pdf at each value of ${\alpha_s}$ at each factorisation and renormalisation scale using FASTNLO tables \cite{FAST}. The ${C_{pdf}}$ was calculated in two different ways for the two different types of pdf. For the NN pdf sets the average value of the cross section, and the standard deviation of each replica from this average was calculated. The ${C_{pdf}}$ matrix was then calculated using equation (\ref{Cov}) where the errors are the standard deviations. 
 
\section{Conclusions}



\begin{acknowledgments}
I would like to thank Daniel Ma\^itre, my supervisor. 
\end{acknowledgments}

\begin{thebibliography}{99}

\bibitem{PPB} PDG, (July 2012), \textit{Particle Physics Booklet.}
\bibitem{DMP}M. Johnson and D. Maitre, (2018), \url{https://arxiv.org/pdf/1711.01408.pdf}, \textit{Strong coupling constant extraction from high-multiplicity Z ${^+}$ jets observables.}
\bibitem{BOOK} M. Peskin and D. Schoeder, (2005), \textit{An Introduction to Quantum Field Theory.}
\bibitem{PHD}Brooks, Helen, Marguerite (2017), \textit{Multi-jet Phenomenology for Hadron Colliders in the High Energy Limit, Durham theses, Durham University. Available at Durham E-Theses Online: http://etheses.dur.ac.uk/12313/.}
\bibitem{GROUP} B. Gutkin, (2016), \textit{Lecture Notes: Group Theory and its applications in physics}
\bibitem{MONTE} A. Larkkoski, (2016), \textit{Lecture Notes on Monte Carlo Methods.}
\bibitem{HEPP} ATLAS Collaboration, (2017), \url{https://arxiv.org/pdf/1702.05725.pdf}, \textit{Measurements of z boson with jets in pp collisions at s=${\sqrt 13}$ TeV.}
\bibitem{HEPD} ATLAS Collaboration (2017), \url{https://www.hepdata.net/record/76542}.
\bibitem{CPDF} S. Alekhin, J. Blumein, and S. Moch, (2012), \url{https://arxiv.org/pdf/1202.2281.pdf}, \textit{Parton distribution functions and benchmark cross sections at NNLO, pg 44.} 
\bibitem{STAT} D. Yoshioka, (2007), \textit{Statistical Physics.}
\bibitem{LUM} W. Herr and B. Muratori, \url{https://cds.cern.ch/record/941318/files/p361.pdf}, \textit{Concept of Luminosity.}
\bibitem{CT10} H.L. Lai, M. Guzzi, J. Huston, Z. Li, P. M. Nadolsky, et al., \url{Phys.Rev. D82, 074024 (2010), arXiv:1007.2241 [hep-ph]}.
\bibitem{CT14} S. Dulat, T.-J. Hou, J. Gao, M. Guzzi, J. Huston, P. Nadolsky, J. Pumplin, C. Schmidt, D. Stump, and C. P. Yuan, \url{Phys. Rev. D93, 033006 (2016), arXiv:1506.07443 [hep-ph].}
\bibitem{MSTW} A. Martin, W. Stirling, R. Thorne, and G. Watt, Eur. Phys. J. C63, 189 (2009), \url{arXiv:0901.0002 [hep-ph]}.
\bibitem{MMHT1} L. A. Harland-Lang, A. D. Martin, P. Motylinski, and R. S. Thorne, Eur. Phys. J. C75, 204 (2015), \url{arXiv:1412.3989 [hep-ph]}.
\bibitem{MMHT2} L. A. Harland-Lang, A. D. Martin, P. Motylinski, and R. S. Thorne, Eur. Phys. J. C75, 435 (2015), \url{arXiv:1506.05682 [hep-ph]}.
\bibitem{NN23} R. D. Ball et al., Nucl. Phys. B867, 244 (2013), \url{arXiv:1207.1303 [hep-ph]}.
\bibitem{NN30} R. D. Ball et al. (NNPDF), JHEP 04, 040 (2015), \url{arXiv:1410.8849 [hep-ph]}.
\bibitem{LHAPDF} \url{https://lhapdf.hepforge.org/pdfsets}
\bibitem{FAST} D. Britzger, K. Rabbertz, F. Stober, and M. Wobisch (2012) (fastNLO), in Proceedings, \url{arXiv:1208.3641 [hep-ph]},\textit{20th Inter-national Workshop on Deep-Inelastic Scattering and Related Subjects (DIS 2012) pp.217?221}.

\end{thebibliography} 


\end{document}